
We have a scalable simulation platform with for \rbc flows through capillaries using boundary integral equations. 
We first presented a robust solver for elliptic \pdes on \threed rigid geometries. 
We thoroughly studied the behavior and performance of this solver on a variety of geometries and compared it with a competitive state-of-the-art solver.
We then parallelized the solver, combined it with boundary integral-based vesicle simulation algorithms and adapted collision-free time stepping to include rigid boundaries.
We have scaled our simulations and the parallel solver to thousands of cores and demonstrated the practicality of using such simulations to reproduce qualitatively representative physical \rbcs flows.

\section{Future Work}
The comparison between \qbkix and \cite{YBZ} in \cref{sec:results-compare} demonstrated the efficiency of a local quadrature scheme compared to a global one. 
Moreover, the scaling results in \cref{ss:scalability} demonstrate that parallel \qbkix is the dominant cost of the simulation.
In order to scale \rbc simulations beyond the regime explored here, the parallel boundary solver needs to be improved. 
A key improvement will be the adpotion of a \textit{local} singular quadrature approach.
Parallel scaling is largely determined by communication costs and \qbkix performs parallel communication entirely through \pvfmm. 
Reducing the number of total points passed to \pvfmm is the best way to improve parallel scaling, since this reduces the overall size of the distributed octree.
The local corrections to an inaccurate \fmm evaluation can be highly vectorized and require no additional parallel communication.
Moreover, the local corrections can be precomputed when solving the integral equation with \gmres. 
These two facts will dramatically increase the performance of \qbkix.

The other main area for improvement is in extrapolation procedure of \qbkix. 
Equally spaced points serve as a fairly bad interpolant, but we have shown that we're able to use low order polynomials to extrapolate reliably.
An important question in the future of \qbkix is how to construct an optimal \oned extrapolation procedure for harmonic functions.
An equally important concern is the scheme's inability to resolve oscillatory \pdes such as the Helmholtz equation.
A trigonometric extrapolation procedure, coupled with a sampling rate comensurate with the solution's underlying frequency, is one possible approach.
