\section{Conclusion\label{sec:conclusion}}
We have presented \qbkix, a fast, high-order, kernel-independent, singular/near-singular quadrature scheme for elliptic boundary value problems in \threed on complex geometries defined by piecewise tensor-product polynomial surfaces.
%We detailed fast algorithms to enforce geometric conditions that ensure accurate singular/near-singular integration throughout the domain.
The primary advantage of our approach is \textit{algorithmic simplicity}: the algorithm can implemented easily with an existing smooth quadrature rule, a point \fmm and \oned and \twod interpolation schemes.
%We presented an error heuristic to trigger upsampling adaptively that incorporates varied surface curvature and is free of Newton iterations.
We presented fast geometry processing algorithms to guarantee accurate singular/near-singular integration, adaptively upsample the discretization and query local surface patches.
We then evaluated \qbkix in various test cases, for Laplace, Stokes, and elasticity problems on various patch-based geometries and compared our approach with \cite{YBZ}.

\cite{lu2019scalable} demonstrates a parallel implementation of \qbkix, but the geometric preprocessing and adaptive upsampling algorithms presented in \cref{sec:algo} are not parallelized.
This is a requirement to solve truly large-scale problems that exist in engineering applications.
Our method can also be easily restructured as a local method.
The comparison in \cref{sec:results-compare} highlights an important point: a local singular quadrature method can outperform a global method for moderate accuracies, \textit{even when the local scheme is asymptotically slower}.
This simple change can also dramatically improve both the serial performance and the parallel scalability of \qbkix shown in \cite{lu2019scalable}, due to the decreased communication of a smaller parallel \fmm evaluation.
The most important improvement to be made, however, is the equispaced extrapolation.
Constructing a superior extrapolation procedure, optimized for the boundary integral context, is the main focus of our current investigations.

\section{Acknowledgements}
We would like to thank Michael O'Neil, Dhairya Malhotra, Libin Lu, Alex Barnett, Leslie Greengard, Michael Shelley for insightful conversations, feedback and suggestions regarding this work. 
We would also like to thank the NYU HPC team, and Shenglong Wang in particular, for great support throughout the course of this work.
This work was supported by NSF grant DMS-1821334.

