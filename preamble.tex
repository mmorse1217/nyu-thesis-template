%% Use the first of the following lines during production to
%% easily spot "overfull boxes" in the output. Use the second
%% line for the final version.
\documentclass[12pt,draft,letterpaper]{report}
%\documentclass[12pt,oneside,letterpaper]{report}


% ----------------------------------------
% Macro to switch between draft version and final version
% ----------------------------------------

% Use or comment this to enable/disable draft version
\def\draftversion{}
\newcommand{\draftfinal}[2]{\ifdefined\draftversion#1\else#2\fi}
\newcommand{\draftonly}[1]{\draftfinal{#1}{}}
\newcommand{\finalonly}[1]{\draftfinal{}{#1}}


% ----------------------------------------
% Thesis metadata
% ----------------------------------------

%% Replace the title, name, advisor name, graduation date and dedication below
%% with your own. Graduation months must be January, May or September.
\newcommand{\thesistitle}{Scalable Blood Flow Simulations with Boundary Integral Equations}
\newcommand{\thesisauthor}{Matthew James Morse}
\newcommand{\thesisadvisor}{Professor Denis Zorin }
\newcommand{\thesisdept}{Computer Science}
\newcommand{\gradmonth}{July}
\newcommand{\gradyear}{2021}
%% If you do not want a dedication, scroll down and comment out
%% the appropriate lines in this file.
\newcommand{\thesisdedication}{To Mom and Dad.}


% ----------------------------------------
% Layout and formatting
% ----------------------------------------

% Uncomment to get a big black box to spot "overfull hboxes"
% \setlength{\overfullrule}{5pt}


%% Page layout (customized to letter paper and NYU requirements):
\RequirePackage[margin=1in, includefoot, letterpaper]{geometry}


%% Color definitions:
\RequirePackage[prologue]{xcolor}
\definecolor[named]{ThesisBlue}{cmyk}{1,0.1,0,0.1}
\definecolor[named]{ThesisYellow}{cmyk}{0,0.16,1,0}
\definecolor[named]{ThesisOrange}{cmyk}{0,0.42,1,0.01}
\definecolor[named]{ThesisRed}{cmyk}{0,0.90,0.86,0}
\definecolor[named]{ThesisLightBlue}{cmyk}{0.49,0.01,0,0}
\definecolor[named]{ThesisGreen}{cmyk}{0.20,0,1,0.19}
\definecolor[named]{ThesisPurple}{cmyk}{0.55,1,0,0.15}
\definecolor[named]{ThesisDarkBlue}{cmyk}{1,0.58,0,0.21}

% School color found from university's graphic identity site:
% http://www.nyu.edu/employees/resources-and-services/media-and-communications/styleguide.html
\definecolor{SchoolColor}{rgb}{0.3412, 0.0235, 0.5490} % purple
\definecolor{chaptergrey}{rgb}{0.2600, 0.0200, 0.4600} % dialed back a little
\definecolor{midgrey}{rgb}{0.4, 0.4, 0.4}

\usepackage{hyperref}
\hypersetup{colorlinks,
  linkcolor=ThesisDarkBlue,
  citecolor=ThesisPurple,
  urlcolor=ThesisDarkBlue,
  filecolor=ThesisDarkBlue}

%% Captions of Figures, tables
\RequirePackage[labelfont={bf,sf,small,singlespacing},
                textfont={sf,small,singlespacing},
                % justification={justified,RaggedRight},
                % singlelinecheck=false,
                margin=0pt,
                figurewithin=chapter,
                tablewithin=chapter]{caption}

%% Chapter headings, captions
\usepackage{fix-cm}
\RequirePackage[raggedright,sc]{titlesec}
\definecolor{gray75}{gray}{0.75}
\newcommand{\hsp}{\hspace{20pt}}

\titleformat{\chapter}[hang]
{\Huge\sc}
{\textcolor{SchoolColor}{\thechapter}\hsp\textcolor{gray75}{|}\hsp}
{0pt}{\Huge\sc\raggedright}
% [\textcolor{gray75}{|}\hsp\textcolor{SchoolColor}{\thechapter}]


%% The following makes chapters and sections, but not subsections,
%% appear in the TOC (table of contents). Increase to 2 or 3 to
%% make subsections or subsubsections appear, respectively. It seems
%% to be usual to use the "1" setting, however.
\setcounter{tocdepth}{2}

%% Sectional units up to subsubsections are numbered. To number
%% subsections, but not subsubsections, decrease this counter to 2.
\setcounter{secnumdepth}{3}

%% Use the following commands, if desired, during production.
%% Comment them out for final version.
%\usepackage{layout} % defines the \layout command, see below
%\setlength{\hoffset}{-.75in} % creates a large right margin for notes and \showlabels

%% Controls spacing between lines (\doublespacing, \onehalfspacing, etc.):
\usepackage{setspace}

%% Use the line below for official NYU version, which requires
%% double line spacing. For all other uses, this is unnecessary,
%% so the line can be commented out.
\finalonly{
  \doublespacing % requires package setspace, invoked above
}

%% For generating sample text.
%% Can be removed when you've replaced all \lipsum commands with your text.
\usepackage{lipsum}


% ----------------------------------------
% Comments and TODOs:
% ----------------------------------------

% Uncomment this to remove all comments
%\newcommand{\nocomments}{}

% Uncomment this to remove all TODOs
%\newcommand{\notodos}{}

% Comments and TODOs
\newcommand{\fcomment}[2]{\ifdefined\nocomments{}\else\footnote{\textcolor{red}{#1:} #2}\fi}
\newcommand{\todo}[1]{\ifdefined\notodos{}\else\textcolor{red}{TODO\ifstrempty{#1}{}{: #1}}\fi}
\newcommand{\ftodo}[1]{\ifdefined\notodos{}\else\fcomment{TODO}{#1}\fi}

% Author comments:
\newcommand{\aen}[1]{\fcomment{Matt}{#1}}


% ----------------------------------------
% User-specific packages and macros
% ----------------------------------------

%% This inputs your auxiliary file with \usepackage's and \newcommand's:
%% It is assumed that that file is called "defs.tex".
% ----------------------------------------
% Packages
% ----------------------------------------

% 
% Place here your \usepackage's. Some recommended packages are already included.
%

% Graphics:
\usepackage[final]{graphicx}
\graphicspath{{figs/}}
%\usepackage{graphicx} % use this line instead of the above to suppress graphics in draft copies
%\usepackage{graphpap} % \defines the \graphpaper command

% Uncomment this to indent first line of each section:
% \usepackage{indentfirst}

% Good AMS stuff:
\usepackage{amsthm} % facilities for theorem-like environments
\usepackage[tbtags]{amsmath} % a lot of good stuff!

% Fonts and symbols:
\usepackage{amsfonts}
\usepackage{amssymb}

% Set the fonts
\RequirePackage[T1]{fontenc}
\ifxetex
  \RequirePackage[tt=false]{libertine}
\else
  \RequirePackage[tt=false, type1=true]{libertine}
\fi
\RequirePackage[varqu]{zi4}
\RequirePackage[libertine]{newtxmath}


% For typesetting inference rules
\usepackage{mathpartir}
% \usepackage{pftools}  % A local package
\newcommand{\bmmax}{2}
\usepackage{bm}

% Formatting tools:
%\usepackage{relsize} % relative font size selection, provides commands \textsmalle, \textlarger
%\usepackage{xspace} % gentle spacing in macros, such as \newcommand{\acims}{\textsc{acim}s\xspace}

% Page formatting utility:
%\usepackage{geometry}

\usepackage{booktabs}   %% For formal tables:
                        %% http://ctan.org/pkg/booktabs
\usepackage[labelformat=simple]{subcaption} %% For complex figures with subfigures/subcaptions
                        %% http://ctan.org/pkg/subcaption
% Options to subcaption are to label and refer to subfigures as Fig 1(a) etc.
\renewcommand\thesubfigure{(\alph{subfigure})}

\usepackage[T1]{fontenc} % needed for scaling fancy fonts (?)
\usepackage[utf8]{inputenc} % not sure this is needed

\usepackage{amssymb}
%\usepackage[table]{xcolor}

% For code
\usepackage[final]{listings}
\lstset{mathescape=true}

% For code highlighting
% \usepackage{bold-extra}

% Tikz
\usepackage{tikz,pgf}
\usetikzlibrary{matrix,arrows,positioning,calc,fit,backgrounds}
\usetikzlibrary{external}
\usetikzlibrary{patterns}
\usepackage{pgfplots,import,tkz-euclide}

% To control enum item labelling/numbering
\usepackage[shortlabels, inline]{enumitem}
% To give custom item labels and reference them
\makeatletter
\newcommand{\myitem}[1][]{
  \protected@edef\@currentlabel{#1}%
\item[#1]
}
\makeatother

% To stop aligned env swallowing up []s
\usepackage{mathtools}

% To use ifstrempty
\usepackage{etoolbox}

% For math mode tables
\usepackage{array}
% A text column in array
\newcolumntype{L}{>$l<$}

% For \llbracket and \rrbracket
\usepackage{stmaryrd}

% For dashed boxes
\usepackage{dashbox}

% For big separating conjunction
\usepackage{scalerel}

% For mathpar environment
\usepackage{mathpartir}

\usepackage{xspace}
\usepackage{multirow}

% To stop citations overflowing lines
\usepackage{breakcites}

% For citet command
\usepackage{natbib}
\setcitestyle{%
    authoryear,%
    open={[},close={]},citesep={;},%
    aysep={},yysep={,},%
    notesep={, }}
\let\cite\citep

%%
%% Place here your \newtheorem's:
%%

\theoremstyle{plain}
\newtheorem{theorem}{Theorem}[chapter]
\newtheorem{conjecture}[theorem]{Conjecture}
\newtheorem{proposition}[theorem]{Proposition}
\newtheorem{lemma}[theorem]{Lemma}
\newtheorem{corollary}[theorem]{Corollary}
\theoremstyle{definition}
\newtheorem{example}[theorem]{Example}
\newtheorem{definition}[theorem]{Definition}
\theoremstyle{plain}


% ----------------------------------------
% Generic definitions
% ----------------------------------------
% Required packages: listings, tikz

% A footnote without a marker
\newcommand\blfootnote[1]{%
  \begingroup
  \renewcommand\thefootnote{}\footnote{#1}%
  \addtocounter{footnote}{-1}%
  \endgroup
}

\renewcommand{\le}{\leqslant}
\renewcommand{\ge}{\geqslant}
% \renewcommand{\emptyset}{\ensuremath{\varnothing}}
% \newcommand{\ds}{\displaystyle}

% Math stuff
\newcommand{\R}{\ensuremath{\mathbb{R}}}
\newcommand{\Q}{\ensuremath{\mathbb{Q}}}
\newcommand{\Z}{\ensuremath{\mathbb{Z}}}
\newcommand{\N}{\ensuremath{\mathbb{N}}}
\newcommand{\T}{\ensuremath{\mathbb{T}}}
\newcommand{\C}{\ensuremath{\mathbb{C}}}
\newcommand{\eps}{\varepsilon}
\newcommand{\closure}[1]{\ensuremath{\overline{#1}}}
%\newcommand{\acim}{\textsc{acim}\xspace}
%\newcommand{\acims}{\textsc{acim}s\xspace}

\newcommand{\Land}{\bigwedge}
\newcommand{\Lor}{\bigvee}
\newcommand{\es}{\emptyset}
\newcommand{\incl}{\subseteq}
\newcommand{\impl}{\Rightarrow}
\renewcommand{\iff}{\Leftrightarrow}
\newcommand{\ra}{\rightarrow}
\newcommand{\sat}{\vDash}
\newcommand{\notsat}{\nvDash}
\newcommand{\proves}{\vdash}
\newcommand{\provesIff}{\mathrel{\dashv\vdash}}
\newcommand{\boolTrue}{\top}
\newcommand{\boolFalse}{\bot}

\newcommand{\dom}{\operatorname{\mathsf{dom}}}
\newcommand{\range}{\operatorname{\mathsf{rng}}}
\newcommand{\restrict}[2]{{#1}|_{#2}}
\newcommand{\pto}{\rightharpoonup}

\newcommand{\defeq}{\coloneqq}
\newcommand{\defiff}{\vcentcolon\iff}

\newcommand{\pipe}{\triangleright}

%% Caligraphic
%\newcommand{\Aa}{{\mathcal{A}}}
%\newcommand{\Bb}{{\mathcal{B}}}
%\newcommand{\Cc}{{\mathcal{C}}}
%\newcommand{\Dd}{{\mathcal{D}}}
%\newcommand{\Ee}{{\mathcal{E}}}
%\newcommand{\Ff}{{\mathcal{F}}}
%\newcommand{\Gg}{{\mathcal{G}}}
%\newcommand{\Hh}{{\mathcal{H}}}
%\newcommand{\Ii}{{\mathcal{I}}}
%\newcommand{\Jj}{{\mathcal{J}}}
%\newcommand{\Kk}{{\mathcal{K}}}
%\newcommand{\Ll}{{\mathcal{L}}}
%\newcommand{\Mm}{{\mathcal{M}}}
%\newcommand{\Nn}{{\mathcal{N}}}
%\newcommand{\Oo}{{\mathcal{O}}}
%\newcommand{\Pp}{{\mathcal{P}}}
%\newcommand{\Qq}{{\mathcal{Q}}}
%\newcommand{\Rr}{{\mathcal{R}}}
%\newcommand{\Ss}{{\mathcal{S}}}
%\newcommand{\Tt}{{\mathcal{T}}}
%\newcommand{\Uu}{{\mathcal{U}}}
%\newcommand{\Vv}{{\mathcal{V}}}
%\newcommand{\Ww}{{\mathcal{W}}}
%\newcommand{\Yy}{{\mathcal{Y}}}
%\newcommand{\Zz}{{\mathcal{Z}}}

% Wrappers: Parens, brackets, etc
% \newcommand{\op}[1]{\operatorname{#1}}
\newcommand{\paren} [1] {\ensuremath{ \left( {#1} \right) }}
\newcommand{\bigparen} [1] {\ensuremath{ \Big( {#1} \Big) }}
% \newcommand{\bracket}[1]{\left[#1\right]}
\newcommand{\tuple}[1]{\ensuremath{\langle #1 \rangle}}
\newcommand{\abs}[1]{\ensuremath{\lvert #1 \rvert}}
% \newcommand{\set}[1]{\ensuremath{\left\{#1\right\}}}
\newcommand{\setcomp}[2]{\ensuremath{\left\{#1\;\middle|\;#2\right\}}}

% References
\newcommand{\refCh}[1]{Chapter~\ref{#1}}
\newcommand{\refSc}[1]{Section~\ref{#1}}
% \newcommand{\refSc}[1]{\S\ref{#1}}
\newcommand{\refFig}[1]{Figure~\ref{#1}}
\newcommand{\refDef}[1]{Definition~\ref{#1}}
\newcommand{\refLem}[1]{Lemma~\ref{#1}}
\newcommand{\refThm}[1]{Theorem~\ref{#1}}
\newcommand{\refAlg}[1]{Algorithm~\ref{#1}}
\newcommand{\refEx}[1]{Example~\ref{#1}}
\newcommand{\refCor}[1]{Corollary~\ref{#1}}
\newcommand{\refTab}[1]{Table~\ref{#1}}
\newcommand{\refEq}[1]{\ensuremath{(\ref{#1})}}
\newcommand{\refRule}[1]{(\ref{#1})}
\newcommand{\refApp}[1]{Appendix~\ref{#1}}

\newcommand{\tool}[1]{\textsf{#1}}
\newcommand{\code}[1]{\textnormal{\small\texttt{#1}}}
% \newcommand{\code}[1]{\text{\lstinline{#1}}}

% TODO have macros for \forall and \exists

\newcommand{\tick}{\ensuremath{\checkmark}}
\newcommand{\cross}{\text{\sffamily X}}


% ----------------------------------------
% Paper specific macros & commands
% ----------------------------------------

%% hedgehog defs
\usepackage{cleveref}
\usepackage[linesnumbered,lined,ruled,vlined,commentsnumbered]{algorithm2e}

\usepackage{enumitem}

\newlist{criteria}{enumerate}{10}
\setlist[criteria]{label*=\arabic*}

\newtheorem{heuristic}[theorem]{Heuristic}
\crefname{heuristic}{heuristic}{heuristics}
\Crefname{heuristic}{Heuristic}{Heuristics}

\crefname{criteriai}{Criterion}{Criteria}
\Crefname{criteriai}{Criterion}{Criteria}
\crefname{alg}{algorithm}{algorithms}
\Crefname{alg}{Algorithm}{Algorithms}
% Put your definitions here
% From Abtin's template
\usepackage{subcaption}
\usepackage{xifthen}
\newcommand{\mcaption}[3]{
        \ifthenelse{\isempty{#2}}
              {\caption[#1]{\small{#3} \label{#1}}}
              {\caption[#2]{{\sc #2.} \small{#3} \label{#1}}}
        }
\newcommand{\algcaption}[3]{
        \ifthenelse{\isempty{#3}}
                   {\caption[#2]{{\sc #2.} \label{#1}}}
                   {\caption[#2]{{\sc #2.} \newline\small{#3} \label{#1}}}
        }

%% bloodflow defs
\def\rbc{\abbrev{RBC}\xspace}
\def\cc{\abbrev{CC}\xspace}
\def\rbcs{\abbrev{RBC}s\xspace}
\def\cpu{\abbrev{CPU}\xspace}
\def\gpu{\abbrev{GPU}\xspace}
\def\mpi{\abbrev{MPI}\xspace}
\def\li{\abbrev{LI}\xspace}
\def\gi{\abbrev{GI}\xspace}
\def\gir{\abbrev{RGI}\xspace}
\def\lic{\abbrev{CLI}\xspace}
\def\lcp{\abbrev{LCP}\xspace}
\def\lcps{\abbrev{LCP}s\xspace}
\def\ncp{\abbrev{NCP}\xspace}
\def\stiv{\abbrev{STIV}\xspace}
\def\p4est{\texttt{p4est}\xspace}
\def\id{\abbrev{ID}\xspace}
\def\ib{\abbrev{IB}\xspace}
\def\lb{\abbrev{LB}\xspace}
\def\dpd{\abbrev{DPD}\xspace}
\def\sph{\abbrev{SPH}\xspace}
\def\skx{\abbrev{SKX}\xspace}
\def\knl{\abbrev{KNL}\xspace}
%\def\sdc{\abbrev{SDC}\xspace}
\newcommand\sdc[1][]{\abbrev{SDC\ifthenelse{\isempty{#1}}{}{\kern 1pt}#1}\xspace}
\def\emdash/{\kern 0.2em---\kern 0.2em}

\newcommand\para[1]{\paragraph*{#1}}
\newcommand\lh[1]{\emph{#1}}           % line heading, useful for itemized

\newcommand\mathbfsf[1]{\bm{\mathsf{#1}}}
\newcommand\vectord[1]{\mathbfsf{#1}}

\newcommand\vB{\vector{B}}
\newcommand\vxd{\vectord{x}}
\newcommand\vXd{\vectord{X}}
\newcommand\vY{\vector{Y}}
\newcommand\vYd{\vectord{Y}}
\newcommand\va{\vector{a}}
\newcommand\vb{\vector{b}}
\newcommand\vf{\vector{f}}
\newcommand\vfd{\vectord{f}}
\newcommand\vF{\vectord{F}}

\newcommand\dt{\ensuremath{\Delta t}}

\newcommand\slyr{single-layer\xspace}
\newcommand\dlyr{double-layer\xspace}
\newcommand\ns{near-singular\xspace}
\newcommand\gl{Gauss--Legendre\xspace}
\newcommand\bc{boundary condition\xspace}

\definecolor{clr1}{RGB}{255, 246, 39}
\definecolor{clr2}{RGB}{124, 22, 28}
\definecolor{clr3}{RGB}{84, 170, 25}
\definecolor{clr4}{RGB}{137, 230, 251}
\definecolor{clr5}{RGB}{226, 49, 39} %red
\definecolor{clr6}{RGB}{12, 59, 136}
\definecolor{clr7}{RGB}{53, 120, 120}
\definecolor{clr8}{RGB}{50, 49, 70}
\definecolor{clr9}{RGB}{255, 0, 255}
\definecolor{clr10}{RGB}{0, 0, 255}
\definecolor{clr11}{RGB}{255, 122, 122}

\definecolor{clr12}{RGB}{130, 130, 130}
\definecolor{clr13}{RGB}{180, 180, 180}
\definecolor{clr14}{RGB}{230, 230, 230}
\definecolor{plt-blue}{rgb}{0.0078,0.2980,0.7961}%blue
\definecolor{plt-orange}{rgb}{1.0000,0.6431,0.2627}%orange
\definecolor{plt-purple}{rgb}{1.0000,0.2863,0.5255}%purple
\definecolor{plt-violet}{rgb}{0.6118,0.1765,1.0000}%violet

\usepackage[export]{adjustbox}  % adjustment boxes in floats
\pgfplotsset{compat=newest}
\usepgfplotslibrary{fillbetween}
\usetikzlibrary{arrows.meta}
\usetikzlibrary{backgrounds}
\usetikzlibrary{pgfplots.groupplots}
\usetikzlibrary{plotmarks}

\newif\ifPlotTikz
\PlotTikztrue
%
\ifPlotTikz
\pgfplotsset{compat=newest}
\usepgfplotslibrary{fillbetween}
\usetikzlibrary{arrows.meta}
\usetikzlibrary{backgrounds}
\usetikzlibrary{pgfplots.groupplots}
\usetikzlibrary{plotmarks}

\pgfplotsset{plot coordinates/math parser=false}
\pgfkeys{/pgf/images/include external/.code=\includegraphics{#1}}
\tikzexternalize[prefix=figs/]
%\tikzset{external/export=false}%disable externalization

\newcommand{\includepgf}[2][1]{
\beginpgfgraphicnamed{#2}%
\tikzsetnextfilename{external-#2}%
\scalebox{#1}{\subimport{figs/}{#2.pgf}}%
\endpgfgraphicnamed%
}

%%disable hyperref in tikz figures
\makeatletter
\tikzset{
    every picture/.style={
        execute at begin picture={
            \let\ref\@refstar
        }
    }
}
\makeatother
%
\else%for ifPlotTikz
%
\newcommand{\includepgf}[2][1]{
\scalebox{#1}{\includegraphics[]{external-#2}}%
}
%
\fi%for ifPlotTikz
%%% Local Variables:
%%% mode: latex
%%% TeX-master: "thesis"
%%% End:

%% macro to define and reference terms with document
\usepackage{relsize}
\newcommand\abbrev[1]{\textsmaller{\uppercase{#1}}} %could do \textsc{\large\lowercase{#1}} or {\small\uppercase{#1}}, insert \textup to keep the letters in Roman/upright form, e.g. \abbrevformat{O\textup{\small pen}MP}

\newcommand\gmres{\abbrev{GMRES}\xspace}
\newcommand\pde{\abbrev{PDE}\xspace}
\newcommand\bvp{\abbrev{BVP}\xspace}
\newcommand\pdes{\abbrev{PDE}s\xspace}
\newcommand\ki{kernel-independent\xspace}
\newcommand\kifmm{\abbrev{KIFMM}\xspace}
\newcommand\fmm{\abbrev{FMM}\,}
\newcommand\fft{\abbrev{FFT}\,}
\newcommand\oned{\abbrev{1D}\,}
\newcommand\twod{\abbrev{2D}\,}
\newcommand\threed{\abbrev{3D}\,}
\newcommand\qbx{\abbrev{QBX}\,}
\newcommand\bie{\abbrev{BIE}\,}
\newcommand\aabb{\abbrev{AABB}\,}
\newcommand\pvfmm{\abbrev{PVFMM}\,}
\newcommand\qbkix{\texttt{hedgehog}\,}
\newcommand\pou{\abbrev{POU}\,}
\newcommand\bem{\abbrev{BEM}\,}
\newcommand\mfs{\abbrev{MFS}\,}
\newcommand\iga{\abbrev{IGA}\,}
\newcommand\wfine{\vector{w}_\text{fine}}
\newcommand\phifine{\vector{\phi}_\text{fine}}
\newcommand\Gammah{\hat{\Gamma}}
\newcommand\nystrom{Nystr\"om\xspace}


%% shortcuts (this paper) %%%%%%%%%%%%%%%%%%%%%%%%%%%%%%%%%%%%%%%%%%%%%%%%%%%%%%%%
\def\GL{Gauss-Legendre}
\def\Pex{\ensuremath{P_\mathrm{ex}}}
\def\Cex{\ensuremath{C_\mathrm{ex}}}
\def\Pnear{\ensuremath{P_\mathrm{near}}}
\def\Cnear{\ensuremath{C_\mathrm{near}}}
\def\Tree{\ensuremath{\mathscr T}}
\def\CenterSet{\ensuremath{\mathscr C}}
\def\PatchSet{\ensuremath{\mathscr P}}
\def\UnPatch{\ensuremath{{\mathscr U}_p}}
\def\UnBox{\ensuremath{{\mathscr U}_b}}
\def\CCover{\ensuremath{C_\mathrm{cover}}}
\def\BCover{\ensuremath{B_\mathrm{cover}}}
\def\Npatch{\ensuremath{N_\mathrm{patch}}}
\def\Ncenter{\ensuremath{N_\mathrm{center}}}
\def\Nbox{\ensuremath{N_\mathrm{box}}}
\DeclareMathOperator{\dist}{dist}


\newcommand\lbl[1]{\ensuremath{\mathrm{#1}}}
\newcommand\qP{{\mathcal{P}}}
\newcommand\I{{\mathcal{I}}}
\newcommand\Pfine{{\mathcal{P}_\lbl{fine}}}
\newcommand\Pcoarse{{\mathcal{P}_\lbl{coarse}}}
\newcommand\err[1]{{\eps_\lbl{#1}}}
\newcommand{\etrg}{\err{target}}
\newcommand\mrm[1]{\mathrm{#1}}
\let\vec\undefined
\let\vector\undefined
\newcommand\vector[1]  {\bm{#1}}
\newcommand\vx{{\vector{x}}}
\newcommand\hr{{\hat{r}}}
\newcommand\vg{{\vector{g}}}
\newcommand\vX{{\vector{X}}}
\newcommand\vP{{\vector{P}}}
\newcommand\vy{{\vector{y}}}
\newcommand\vd{{\vector{d}}}
\newcommand\vc{{\vector{c}}}
\newcommand\vhc{{\hat{\vector{c}}}}
\newcommand\vz{{\vector{z}}}
\newcommand\vn{{\vector{n}}}
\newcommand\vr{{\vector{r}}}
\newcommand\vsx{{\vector{s}_{\vector{x}}}}
\newcommand\vu{{\vector{u}}}
\newcommand\Cone{{C_{(1)}}}
\newcommand\Ctwo{{C_{(2)}}}
\DeclareMathOperator*{\argmin}{arg\,min}
\newcommand{\RN}[1]{%
    \textup{\uppercase\expandafter{\romannumeral#1}}%
  }
\usepackage{mathtools}
\DeclarePairedDelimiter\ceil{\lceil}{\rceil}
\DeclarePairedDelimiter\floor{\lfloor}{\rfloor}

\newcommand\Nf{\mathcal{N}_\lbl{far}}
\newcommand\Ni{\mathcal{N}_\lbl{inter}}
\newcommand\Nn{\mathcal{N}_\lbl{near}}
\newcommand\Nt{\mathcal{N}_\lbl{tot}}
\newcommand\Ninit{N_\lbl{init}}

\newcommand{\Lmx}{L_{\lbl{max}}}
\newcommand{\Lmn}{L_{\lbl{min}}}
\newcommand{\ellm}{\ell_{min}}




% ----------------------------------------
% Document header
% ----------------------------------------

%% Cross-referencing utilities. Use one or the other--whichever you prefer--
%% but comment out both lines for final version.
%\usepackage{showlabels}
%\usepackage{showkeys}


