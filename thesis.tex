%% ----------------------------------------
%%
%% NYU PhD thesis template.
%% Created by José Koiller 2007--2008.
%% Modified by Siddharth Krishna, 2019.
%%
%% ----------------------------------------


%% Use the first of the following lines during production to
%% easily spot "overfull boxes" in the output. Use the second
%% line for the final version.
%\documentclass[12pt,draft,letterpaper]{report}
\documentclass[12pt,oneside,letterpaper]{report}


% ----------------------------------------
% Macro to switch between draft version and final version
% ----------------------------------------

% Use or comment this to enable/disable draft version
% \def\draftversion{}
\newcommand{\draftfinal}[2]{\ifdefined\draftversion#1\else#2\fi}
\newcommand{\draftonly}[1]{\draftfinal{#1}{}}
\newcommand{\finalonly}[1]{\draftfinal{}{#1}}


% ----------------------------------------
% Thesis metadata
% ----------------------------------------

%% Replace the title, name, advisor name, graduation date and dedication below
%% with your own. Graduation months must be January, May or September.
\newcommand{\thesistitle}{On Complete Systems of Invariants for Ternary Biquadratic Forms}
\newcommand{\thesisauthor}{Amalie Emmy Noether}
\newcommand{\thesisadvisor}{Professor Paul Gordan}
\newcommand{\thesisdept}{Mathematics}
\newcommand{\gradmonth}{September}
\newcommand{\gradyear}{2019}
%% If you do not want a dedication, scroll down and comment out
%% the appropriate lines in this file.
\newcommand{\thesisdedication}{To my dog Weierstra\ss, with affection.}


% ----------------------------------------
% Layout and formatting
% ----------------------------------------

% Uncomment to get a big black box to spot "overfull hboxes"
% \setlength{\overfullrule}{5pt}


%% Page layout (customized to letter paper and NYU requirements):
\RequirePackage[margin=1in, includefoot, letterpaper]{geometry}


%% Color definitions:
\RequirePackage[prologue]{xcolor}
\definecolor[named]{ThesisBlue}{cmyk}{1,0.1,0,0.1}
\definecolor[named]{ThesisYellow}{cmyk}{0,0.16,1,0}
\definecolor[named]{ThesisOrange}{cmyk}{0,0.42,1,0.01}
\definecolor[named]{ThesisRed}{cmyk}{0,0.90,0.86,0}
\definecolor[named]{ThesisLightBlue}{cmyk}{0.49,0.01,0,0}
\definecolor[named]{ThesisGreen}{cmyk}{0.20,0,1,0.19}
\definecolor[named]{ThesisPurple}{cmyk}{0.55,1,0,0.15}
\definecolor[named]{ThesisDarkBlue}{cmyk}{1,0.58,0,0.21}

% School color found from university's graphic identity site:
% http://www.nyu.edu/employees/resources-and-services/media-and-communications/styleguide.html
\definecolor{SchoolColor}{rgb}{0.3412, 0.0235, 0.5490} % purple
\definecolor{chaptergrey}{rgb}{0.2600, 0.0200, 0.4600} % dialed back a little
\definecolor{midgrey}{rgb}{0.4, 0.4, 0.4}

\usepackage{hyperref}
\hypersetup{colorlinks,
  linkcolor=ThesisDarkBlue,
  citecolor=ThesisPurple,
  urlcolor=ThesisDarkBlue,
  filecolor=ThesisDarkBlue}


%% Captions of Figures, tables
\RequirePackage[labelfont={bf,sf,small,singlespacing},
                textfont={sf,small,singlespacing},
                % justification={justified,RaggedRight},
                % singlelinecheck=false,
                margin=0pt,
                figurewithin=chapter,
                tablewithin=chapter]{caption}

%% Chapter headings, captions
\usepackage{fix-cm}
\RequirePackage[raggedright,sc]{titlesec}
\definecolor{gray75}{gray}{0.75}
\newcommand{\hsp}{\hspace{20pt}}

\titleformat{\chapter}[hang]
{\Huge\sc}
{\textcolor{SchoolColor}{\thechapter}\hsp\textcolor{gray75}{|}\hsp}
{0pt}{\Huge\sc\raggedright}
% [\textcolor{gray75}{|}\hsp\textcolor{SchoolColor}{\thechapter}]


%% The following makes chapters and sections, but not subsections,
%% appear in the TOC (table of contents). Increase to 2 or 3 to
%% make subsections or subsubsections appear, respectively. It seems
%% to be usual to use the "1" setting, however.
\setcounter{tocdepth}{2}

%% Sectional units up to subsubsections are numbered. To number
%% subsections, but not subsubsections, decrease this counter to 2.
\setcounter{secnumdepth}{3}

%% Use the following commands, if desired, during production.
%% Comment them out for final version.
%\usepackage{layout} % defines the \layout command, see below
%\setlength{\hoffset}{-.75in} % creates a large right margin for notes and \showlabels

%% Controls spacing between lines (\doublespacing, \onehalfspacing, etc.):
\usepackage{setspace}

%% Use the line below for official NYU version, which requires
%% double line spacing. For all other uses, this is unnecessary,
%% so the line can be commented out.
\finalonly{
  \doublespacing % requires package setspace, invoked above
}

%% For generating sample text.
%% Can be removed when you've replaced all \lipsum commands with your text.
\usepackage{lipsum}


% ----------------------------------------
% Comments and TODOs:
% ----------------------------------------

% Uncomment this to remove all comments
\newcommand{\nocomments}{}

% Uncomment this to remove all TODOs
\newcommand{\notodos}{}

% Comments and TODOs
\newcommand{\fcomment}[2]{\ifdefined\nocomments{}\else\footnote{\textcolor{red}{#1:} #2}\fi}
\newcommand{\todo}[1]{\ifdefined\notodos{}\else\textcolor{red}{TODO\ifstrempty{#1}{}{: #1}}\fi}
\newcommand{\ftodo}[1]{\ifdefined\notodos{}\else\fcomment{TODO}{#1}\fi}

% Author comments:
\newcommand{\aen}[1]{\fcomment{Emmy}{#1}}


% ----------------------------------------
% User-specific packages and macros
% ----------------------------------------

%% This inputs your auxiliary file with \usepackage's and \newcommand's:
%% It is assumed that that file is called "defs.tex".
% ----------------------------------------
% Packages
% ----------------------------------------

% 
% Place here your \usepackage's. Some recommended packages are already included.
%

% Graphics:
\usepackage[final]{graphicx}
\graphicspath{{figs/}}
%\usepackage{graphicx} % use this line instead of the above to suppress graphics in draft copies
%\usepackage{graphpap} % \defines the \graphpaper command

% Uncomment this to indent first line of each section:
% \usepackage{indentfirst}

% Good AMS stuff:
\usepackage{amsthm} % facilities for theorem-like environments
\usepackage[tbtags]{amsmath} % a lot of good stuff!

% Fonts and symbols:
\usepackage{amsfonts}
\usepackage{amssymb}

% Set the fonts
\RequirePackage[T1]{fontenc}
\ifxetex
  \RequirePackage[tt=false]{libertine}
\else
  \RequirePackage[tt=false, type1=true]{libertine}
\fi
\RequirePackage[varqu]{zi4}
\RequirePackage[libertine]{newtxmath}


% For typesetting inference rules
\usepackage{mathpartir}
% \usepackage{pftools}  % A local package
\newcommand{\bmmax}{2}
\usepackage{bm}

% Formatting tools:
%\usepackage{relsize} % relative font size selection, provides commands \textsmalle, \textlarger
%\usepackage{xspace} % gentle spacing in macros, such as \newcommand{\acims}{\textsc{acim}s\xspace}

% Page formatting utility:
%\usepackage{geometry}

\usepackage{booktabs}   %% For formal tables:
                        %% http://ctan.org/pkg/booktabs
\usepackage[labelformat=simple]{subcaption} %% For complex figures with subfigures/subcaptions
                        %% http://ctan.org/pkg/subcaption
% Options to subcaption are to label and refer to subfigures as Fig 1(a) etc.
\renewcommand\thesubfigure{(\alph{subfigure})}

\usepackage[T1]{fontenc} % needed for scaling fancy fonts (?)
\usepackage[utf8]{inputenc} % not sure this is needed

\usepackage{amssymb}
%\usepackage[table]{xcolor}

% For code
\usepackage[final]{listings}
\lstset{mathescape=true}

% For code highlighting
% \usepackage{bold-extra}

% Tikz
\usepackage{tikz,pgf}
\usetikzlibrary{matrix,arrows,positioning,calc,fit,backgrounds}
\usetikzlibrary{external}
\usetikzlibrary{patterns}
\usepackage{pgfplots,import,tkz-euclide}

% To control enum item labelling/numbering
\usepackage[shortlabels, inline]{enumitem}
% To give custom item labels and reference them
\makeatletter
\newcommand{\myitem}[1][]{
  \protected@edef\@currentlabel{#1}%
\item[#1]
}
\makeatother

% To stop aligned env swallowing up []s
\usepackage{mathtools}

% To use ifstrempty
\usepackage{etoolbox}

% For math mode tables
\usepackage{array}
% A text column in array
\newcolumntype{L}{>$l<$}

% For \llbracket and \rrbracket
\usepackage{stmaryrd}

% For dashed boxes
\usepackage{dashbox}

% For big separating conjunction
\usepackage{scalerel}

% For mathpar environment
\usepackage{mathpartir}

\usepackage{xspace}
\usepackage{multirow}

% To stop citations overflowing lines
\usepackage{breakcites}

% For citet command
\usepackage{natbib}
\setcitestyle{%
    authoryear,%
    open={[},close={]},citesep={;},%
    aysep={},yysep={,},%
    notesep={, }}
\let\cite\citep

%%
%% Place here your \newtheorem's:
%%

\theoremstyle{plain}
\newtheorem{theorem}{Theorem}[chapter]
\newtheorem{conjecture}[theorem]{Conjecture}
\newtheorem{proposition}[theorem]{Proposition}
\newtheorem{lemma}[theorem]{Lemma}
\newtheorem{corollary}[theorem]{Corollary}
\theoremstyle{definition}
\newtheorem{example}[theorem]{Example}
\newtheorem{definition}[theorem]{Definition}
\theoremstyle{plain}


% ----------------------------------------
% Generic definitions
% ----------------------------------------
% Required packages: listings, tikz

% A footnote without a marker
\newcommand\blfootnote[1]{%
  \begingroup
  \renewcommand\thefootnote{}\footnote{#1}%
  \addtocounter{footnote}{-1}%
  \endgroup
}

\renewcommand{\le}{\leqslant}
\renewcommand{\ge}{\geqslant}
% \renewcommand{\emptyset}{\ensuremath{\varnothing}}
% \newcommand{\ds}{\displaystyle}

% Math stuff
\newcommand{\R}{\ensuremath{\mathbb{R}}}
\newcommand{\Q}{\ensuremath{\mathbb{Q}}}
\newcommand{\Z}{\ensuremath{\mathbb{Z}}}
\newcommand{\N}{\ensuremath{\mathbb{N}}}
\newcommand{\T}{\ensuremath{\mathbb{T}}}
\newcommand{\C}{\ensuremath{\mathbb{C}}}
\newcommand{\eps}{\varepsilon}
\newcommand{\closure}[1]{\ensuremath{\overline{#1}}}
%\newcommand{\acim}{\textsc{acim}\xspace}
%\newcommand{\acims}{\textsc{acim}s\xspace}

\newcommand{\Land}{\bigwedge}
\newcommand{\Lor}{\bigvee}
\newcommand{\es}{\emptyset}
\newcommand{\incl}{\subseteq}
\newcommand{\impl}{\Rightarrow}
\renewcommand{\iff}{\Leftrightarrow}
\newcommand{\ra}{\rightarrow}
\newcommand{\sat}{\vDash}
\newcommand{\notsat}{\nvDash}
\newcommand{\proves}{\vdash}
\newcommand{\provesIff}{\mathrel{\dashv\vdash}}
\newcommand{\boolTrue}{\top}
\newcommand{\boolFalse}{\bot}

\newcommand{\dom}{\operatorname{\mathsf{dom}}}
\newcommand{\range}{\operatorname{\mathsf{rng}}}
\newcommand{\restrict}[2]{{#1}|_{#2}}
\newcommand{\pto}{\rightharpoonup}

\newcommand{\defeq}{\coloneqq}
\newcommand{\defiff}{\vcentcolon\iff}

\newcommand{\pipe}{\triangleright}

%% Caligraphic
%\newcommand{\Aa}{{\mathcal{A}}}
%\newcommand{\Bb}{{\mathcal{B}}}
%\newcommand{\Cc}{{\mathcal{C}}}
%\newcommand{\Dd}{{\mathcal{D}}}
%\newcommand{\Ee}{{\mathcal{E}}}
%\newcommand{\Ff}{{\mathcal{F}}}
%\newcommand{\Gg}{{\mathcal{G}}}
%\newcommand{\Hh}{{\mathcal{H}}}
%\newcommand{\Ii}{{\mathcal{I}}}
%\newcommand{\Jj}{{\mathcal{J}}}
%\newcommand{\Kk}{{\mathcal{K}}}
%\newcommand{\Ll}{{\mathcal{L}}}
%\newcommand{\Mm}{{\mathcal{M}}}
%\newcommand{\Nn}{{\mathcal{N}}}
%\newcommand{\Oo}{{\mathcal{O}}}
%\newcommand{\Pp}{{\mathcal{P}}}
%\newcommand{\Qq}{{\mathcal{Q}}}
%\newcommand{\Rr}{{\mathcal{R}}}
%\newcommand{\Ss}{{\mathcal{S}}}
%\newcommand{\Tt}{{\mathcal{T}}}
%\newcommand{\Uu}{{\mathcal{U}}}
%\newcommand{\Vv}{{\mathcal{V}}}
%\newcommand{\Ww}{{\mathcal{W}}}
%\newcommand{\Yy}{{\mathcal{Y}}}
%\newcommand{\Zz}{{\mathcal{Z}}}

% Wrappers: Parens, brackets, etc
% \newcommand{\op}[1]{\operatorname{#1}}
\newcommand{\paren} [1] {\ensuremath{ \left( {#1} \right) }}
\newcommand{\bigparen} [1] {\ensuremath{ \Big( {#1} \Big) }}
% \newcommand{\bracket}[1]{\left[#1\right]}
\newcommand{\tuple}[1]{\ensuremath{\langle #1 \rangle}}
\newcommand{\abs}[1]{\ensuremath{\lvert #1 \rvert}}
% \newcommand{\set}[1]{\ensuremath{\left\{#1\right\}}}
\newcommand{\setcomp}[2]{\ensuremath{\left\{#1\;\middle|\;#2\right\}}}

% References
\newcommand{\refCh}[1]{Chapter~\ref{#1}}
\newcommand{\refSc}[1]{Section~\ref{#1}}
% \newcommand{\refSc}[1]{\S\ref{#1}}
\newcommand{\refFig}[1]{Figure~\ref{#1}}
\newcommand{\refDef}[1]{Definition~\ref{#1}}
\newcommand{\refLem}[1]{Lemma~\ref{#1}}
\newcommand{\refThm}[1]{Theorem~\ref{#1}}
\newcommand{\refAlg}[1]{Algorithm~\ref{#1}}
\newcommand{\refEx}[1]{Example~\ref{#1}}
\newcommand{\refCor}[1]{Corollary~\ref{#1}}
\newcommand{\refTab}[1]{Table~\ref{#1}}
\newcommand{\refEq}[1]{\ensuremath{(\ref{#1})}}
\newcommand{\refRule}[1]{(\ref{#1})}
\newcommand{\refApp}[1]{Appendix~\ref{#1}}

\newcommand{\tool}[1]{\textsf{#1}}
\newcommand{\code}[1]{\textnormal{\small\texttt{#1}}}
% \newcommand{\code}[1]{\text{\lstinline{#1}}}

% TODO have macros for \forall and \exists

\newcommand{\tick}{\ensuremath{\checkmark}}
\newcommand{\cross}{\text{\sffamily X}}


% ----------------------------------------
% Paper specific macros & commands
% ----------------------------------------

%% hedgehog defs
\usepackage{cleveref}
\usepackage[linesnumbered,lined,ruled,vlined,commentsnumbered]{algorithm2e}

\usepackage{enumitem}

\newlist{criteria}{enumerate}{10}
\setlist[criteria]{label*=\arabic*}

\newtheorem{heuristic}[theorem]{Heuristic}
\crefname{heuristic}{heuristic}{heuristics}
\Crefname{heuristic}{Heuristic}{Heuristics}

\crefname{criteriai}{Criterion}{Criteria}
\Crefname{criteriai}{Criterion}{Criteria}
\crefname{alg}{algorithm}{algorithms}
\Crefname{alg}{Algorithm}{Algorithms}
% Put your definitions here
% From Abtin's template
\usepackage{subcaption}
\usepackage{xifthen}
\newcommand{\mcaption}[3]{
        \ifthenelse{\isempty{#2}}
              {\caption[#1]{\small{#3} \label{#1}}}
              {\caption[#2]{{\sc #2.} \small{#3} \label{#1}}}
        }
\newcommand{\algcaption}[3]{
        \ifthenelse{\isempty{#3}}
                   {\caption[#2]{{\sc #2.} \label{#1}}}
                   {\caption[#2]{{\sc #2.} \newline\small{#3} \label{#1}}}
        }

%% bloodflow defs
\def\rbc{\abbrev{RBC}\xspace}
\def\cc{\abbrev{CC}\xspace}
\def\rbcs{\abbrev{RBC}s\xspace}
\def\cpu{\abbrev{CPU}\xspace}
\def\gpu{\abbrev{GPU}\xspace}
\def\mpi{\abbrev{MPI}\xspace}
\def\li{\abbrev{LI}\xspace}
\def\gi{\abbrev{GI}\xspace}
\def\gir{\abbrev{RGI}\xspace}
\def\lic{\abbrev{CLI}\xspace}
\def\lcp{\abbrev{LCP}\xspace}
\def\lcps{\abbrev{LCP}s\xspace}
\def\ncp{\abbrev{NCP}\xspace}
\def\stiv{\abbrev{STIV}\xspace}
\def\p4est{\texttt{p4est}\xspace}
\def\id{\abbrev{ID}\xspace}
\def\ib{\abbrev{IB}\xspace}
\def\lb{\abbrev{LB}\xspace}
\def\dpd{\abbrev{DPD}\xspace}
\def\sph{\abbrev{SPH}\xspace}
\def\skx{\abbrev{SKX}\xspace}
\def\knl{\abbrev{KNL}\xspace}
%\def\sdc{\abbrev{SDC}\xspace}
\newcommand\sdc[1][]{\abbrev{SDC\ifthenelse{\isempty{#1}}{}{\kern 1pt}#1}\xspace}
\def\emdash/{\kern 0.2em---\kern 0.2em}

\newcommand\para[1]{\paragraph*{#1}}
\newcommand\lh[1]{\emph{#1}}           % line heading, useful for itemized

\newcommand\mathbfsf[1]{\bm{\mathsf{#1}}}
\newcommand\vectord[1]{\mathbfsf{#1}}

\newcommand\vB{\vector{B}}
\newcommand\vxd{\vectord{x}}
\newcommand\vXd{\vectord{X}}
\newcommand\vY{\vector{Y}}
\newcommand\vYd{\vectord{Y}}
\newcommand\va{\vector{a}}
\newcommand\vb{\vector{b}}
\newcommand\vf{\vector{f}}
\newcommand\vfd{\vectord{f}}
\newcommand\vF{\vectord{F}}

\newcommand\dt{\ensuremath{\Delta t}}

\newcommand\slyr{single-layer\xspace}
\newcommand\dlyr{double-layer\xspace}
\newcommand\ns{near-singular\xspace}
\newcommand\gl{Gauss--Legendre\xspace}
\newcommand\bc{boundary condition\xspace}

\definecolor{clr1}{RGB}{255, 246, 39}
\definecolor{clr2}{RGB}{124, 22, 28}
\definecolor{clr3}{RGB}{84, 170, 25}
\definecolor{clr4}{RGB}{137, 230, 251}
\definecolor{clr5}{RGB}{226, 49, 39} %red
\definecolor{clr6}{RGB}{12, 59, 136}
\definecolor{clr7}{RGB}{53, 120, 120}
\definecolor{clr8}{RGB}{50, 49, 70}
\definecolor{clr9}{RGB}{255, 0, 255}
\definecolor{clr10}{RGB}{0, 0, 255}
\definecolor{clr11}{RGB}{255, 122, 122}

\definecolor{clr12}{RGB}{130, 130, 130}
\definecolor{clr13}{RGB}{180, 180, 180}
\definecolor{clr14}{RGB}{230, 230, 230}
\definecolor{plt-blue}{rgb}{0.0078,0.2980,0.7961}%blue
\definecolor{plt-orange}{rgb}{1.0000,0.6431,0.2627}%orange
\definecolor{plt-purple}{rgb}{1.0000,0.2863,0.5255}%purple
\definecolor{plt-violet}{rgb}{0.6118,0.1765,1.0000}%violet

\usepackage[export]{adjustbox}  % adjustment boxes in floats
\pgfplotsset{compat=newest}
\usepgfplotslibrary{fillbetween}
\usetikzlibrary{arrows.meta}
\usetikzlibrary{backgrounds}
\usetikzlibrary{pgfplots.groupplots}
\usetikzlibrary{plotmarks}

\newif\ifPlotTikz
\PlotTikztrue
%
\ifPlotTikz
\pgfplotsset{compat=newest}
\usepgfplotslibrary{fillbetween}
\usetikzlibrary{arrows.meta}
\usetikzlibrary{backgrounds}
\usetikzlibrary{pgfplots.groupplots}
\usetikzlibrary{plotmarks}

\pgfplotsset{plot coordinates/math parser=false}
\pgfkeys{/pgf/images/include external/.code=\includegraphics{#1}}
\tikzexternalize[prefix=figs/]
%\tikzset{external/export=false}%disable externalization

\newcommand{\includepgf}[2][1]{
\beginpgfgraphicnamed{#2}%
\tikzsetnextfilename{external-#2}%
\scalebox{#1}{\subimport{figs/}{#2.pgf}}%
\endpgfgraphicnamed%
}

%%disable hyperref in tikz figures
\makeatletter
\tikzset{
    every picture/.style={
        execute at begin picture={
            \let\ref\@refstar
        }
    }
}
\makeatother
%
\else%for ifPlotTikz
%
\newcommand{\includepgf}[2][1]{
\scalebox{#1}{\includegraphics[]{external-#2}}%
}
%
\fi%for ifPlotTikz
%%% Local Variables:
%%% mode: latex
%%% TeX-master: "thesis"
%%% End:



% ----------------------------------------
% Document header
% ----------------------------------------

%% Cross-referencing utilities. Use one or the other--whichever you prefer--
%% but comment out both lines for final version.
%\usepackage{showlabels}
%\usepackage{showkeys}

\begin{document}
%% Produces a test "layout" page, for "debugging" purposes only.
%% Comment out for final version.
%\layout % requires package layout (see above, on this same file)


%%%%%% Title page %%%%%%%%%%%
%% Sets page numbering to "roman style" i, ii, iii, iv, etc:
\pagenumbering{roman}
%
%% No numbering in the title page:
\thispagestyle{empty}
%
\vspace*{25pt}
\begin{center}
  {\Large
    \begin{doublespace}
      {\textcolor{SchoolColor}{\textsc{\thesistitle}}}
    \end{doublespace}
  }
  \vspace{.7in}

  by
  \vspace{.7in}

  \thesisauthor
  \vfill

  \begin{doublespace}
    \textsc{
    A dissertation submitted in partial fulfillment\\
    of the requirements for the degree of\\
    Doctor of Philosophy\\
    Department of \thesisdept\\
    New York University\\
    \gradmonth, \gradyear}
  \end{doublespace}
\end{center}
\vfill

\noindent\makebox[\textwidth]{\hfill\makebox[2.5in]{\hrulefill}}\\
\makebox[\textwidth]{\hfill\makebox[2.5in]{\hfill\thesisadvisor}}

\newpage


%%%%%%%%%%%%% Copyright page %%%%%%%%%%%%%%%%%%
\thispagestyle{empty}
\vspace*{25pt}
\begin{center}
  \scshape \noindent \small \copyright \  \small  \thesisauthor \\
  all rights reserved, \gradyear
\end{center}
\vspace*{0in}
\newpage


%%%%%%%%%%%%%% Dedication %%%%%%%%%%%%%%%%%
%% Comment out the following lines if you do not want to dedicate
%% this to anyone...
\cleardoublepage
\phantomsection
\addcontentsline{toc}{chapter}{Dedication}
\vspace*{\fill}
\begin{center}
  \thesisdedication
\end{center}
\vfill
\newpage


%%%%%%%%%%%%%% Acknowledgements %%%%%%%%%%%%
%% Comment out the following lines if you do not want to acknowledge
%% anyone's help...
\chapter*{Acknowledgements}
\addcontentsline{toc}{chapter}{Acknowledgments}

% \input{acknowledge}
\lipsum[1-2]

\newpage


%%%% Abstract %%%%%%%%%%%%%%%%%%
\chapter*{Abstract}
\addcontentsline{toc}{chapter}{Abstract}

% Replace this with your abstract:
Abstract here...

%%% Local Variables:
%%% mode: latex
%%% TeX-master: "thesis"
%%% End:


\newpage


%%%% Table of Contents %%%%%%%%%%%%
\tableofcontents


%%%%% List of Figures %%%%%%%%%%%%%
%% Comment out the following two lines if your thesis does not
%% contain any figures. The list of figures contains only
%% those figures included within the "figure" environment.
\cleardoublepage
\phantomsection
\addcontentsline{toc}{chapter}{List of Figures}
\listoffigures
\newpage


%%%%% List of Tables %%%%%%%%%%%%%
%% Comment out the following two lines if your thesis does not
%% contain any tables. The list of tables contains only
%% those tables included within the "table" environment.
\cleardoublepage
\phantomsection
\addcontentsline{toc}{chapter}{List of Tables}
\listoftables
\newpage


%%%%% Body of thesis starts %%%%%%%%%%%%
\pagenumbering{arabic} % switches page numbering to arabic: 1, 2, 3, etc.


% ----------------------------------------
% Body of Thesis
% ----------------------------------------

\chapter{Introduction}
\label{chp-introduction}

% \input{introduction}

% Sample content:
\lipsum[1-2]

\begin{figure}
  \centering
  \begin{tikzpicture}[scale=3]
    \draw[step=.5cm, gray, very thin] (-1.2,-1.2) grid (1.2,1.2); 
    \filldraw[fill=green!20,draw=green!50!black] (0,0) -- (3mm,0mm) arc (0:30:3mm) -- cycle; 
    \draw[->] (-1.25,0) -- (1.25,0) coordinate (x axis);
    \draw[->] (0,-1.25) -- (0,1.25) coordinate (y axis);
    \draw (0,0) circle (1cm);
    \draw[very thick,red] (30:1cm) -- node[left,fill=white] {$\sin \alpha$} (30:1cm |- x axis);
    \draw[very thick,blue] (30:1cm |- x axis) -- node[below=2pt,fill=white] {$\cos \alpha$} (0,0);
    \draw (0,0) -- (30:1cm);
    \foreach \x/\xtext in {-1, -0.5/-\frac{1}{2}, 1} 
    \draw (\x cm,1pt) -- (\x cm,-1pt) node[anchor=north,fill=white] {$\xtext$};
    \foreach \y/\ytext in {-1, -0.5/-\frac{1}{2}, 0.5/\frac{1}{2}, 1} 
    \draw (1pt,\y cm) -- (-1pt,\y cm) node[anchor=east,fill=white] {$\ytext$};
  \end{tikzpicture}
  \caption{A pictorial view of \refThm{thm-pythagorean}.}
  \label{fig-pythagorean}
\end{figure}

\begin{definition}
  A function $f$ is said to be \emph{continuous} if its derivative exists at every point.
\end{definition}

\begin{lemma}
  Let $f$ be a function whose derivative exists in every point, then $f$ is 
  a continuous function.
\end{lemma}

\begin{theorem}[Pythagorean theorem]
  \label{thm-pythagorean}
  This is a theorema about right triangles and can be summarised in the next 
  equation 
  \[ x^2 + y^2 = z^2 \]
\end{theorem}
\begin{proof}
  I have discovered a truly marvelous proof of this, which this margin is too narrow to contain.
\end{proof}

And a consequence of \refThm{thm-pythagorean} is the statement in the next 
corollary~\cite{lamport94}.

\begin{corollary}
  There's no right rectangle whose sides measure 3cm, 4cm, and 6cm.
\end{corollary}

\lipsum[3-5]

\begin{table}
  \centering
  \caption{Predicted final standings of Group B.}
  \begin{tabular}{l*{6}{c}r}
    Team              & P & W & D & L & F  & A & Pts \\
    \hline
    Manchester United & 6 & 4 & 0 & 2 & 10 & 5 & 12  \\
    Celtic            & 6 & 3 & 0 & 3 &  8 & 9 &  9  \\
    Benfica           & 6 & 2 & 1 & 3 &  7 & 8 &  7  \\
    FC Copenhagen     & 6 & 2 & 1 & 3 &  5 & 8 &  7  \\
  \end{tabular}
  \label{tab-forecast}
\end{table}

\lipsum[5-7]

\chapter{Preliminaries}
\label{chp-preliminaries}

% \input{preliminaries}
\lipsum

\chapter{Proof of the Reimann Hypothesis}
\label{chp-proof}

% \input{proof}
\lipsum

\chapter{Conclusion}
\label{chp-conclusion}

% \section{Conclusion\label{sec:conclusion}}
We have presented \qbkix, a fast, high-order, kernel-independent, singular/near-singular quadrature scheme for elliptic boundary value problems in \threed on complex geometries defined by piecewise tensor-product polynomial surfaces.
%We detailed fast algorithms to enforce geometric conditions that ensure accurate singular/near-singular integration throughout the domain.
The primary advantage of our approach is \textit{algorithmic simplicity}: the algorithm can implemented easily with an existing smooth quadrature rule, a point \fmm and \oned and \twod interpolation schemes.
%We presented an error heuristic to trigger upsampling adaptively that incorporates varied surface curvature and is free of Newton iterations.
We presented fast geometry processing algorithms to guarantee accurate singular/near-singular integration, adaptively upsample the discretization and query local surface patches.
We then evaluated \qbkix in various test cases, for Laplace, Stokes, and elasticity problems on various patch-based geometries and compared our approach with \cite{YBZ}.

\cite{lu2019scalable} demonstrates a parallel implementation of \qbkix, but the geometric preprocessing and adaptive upsampling algorithms presented in \cref{sec:algo} are not parallelized.
This is a requirement to solve truly large-scale problems that exist in engineering applications.
Our method can also be easily restructured as a local method.
The comparison in \cref{sec:results-compare} highlights an important point: a local singular quadrature method can outperform a global method for moderate accuracies, \textit{even when the local scheme is asymptotically slower}.
This simple change can also dramatically improve both the serial performance and the parallel scalability of \qbkix shown in \cite{lu2019scalable}, due to the decreased communication of a smaller parallel \fmm evaluation.
The most important improvement to be made, however, is the equispaced extrapolation.
Constructing a superior extrapolation procedure, optimized for the boundary integral context, is the main focus of our current investigations.

\iffalse
\section{Acknowledgements}
We would like to thank Michael O'Neil, Dhairya Malhotra, Libin Lu, Alex Barnett, Leslie Greengard, Michael Shelley for insightful conversations, feedback and suggestions regarding this work. 
We would also like to thank the NYU HPC team, and Shenglong Wang in particular, for great support throughout the course of this work.
This work was supported by NSF grant DMS-1821334.
\fi

\lipsum


%% If your thesis has different "Parts", use commands such as the following:
%\part{First Part\label{part:one}}%
% \input{chap1}
%\input{chap2} % further chapters -- change file names to meaningful things...
%\input{chap3}
%\part{Second Part\label{part:two}}%
%\input{chap4}
%\input{chap5}
%\input{chap6}


%%%%% Appendices start %%%%%%%%%%%%%%%%
%% Comment out the following if your thesis has no appendix

\appendix

\chapter{Appendix}

% 
\section{Kernels\label{app:kernels}}
Here we list the elliptic \pde's investigated in this work along with the associated kernels for their single- and double-layer potentials.
In this section, $\vx$ and $\vy$ are in $\mathbb{R}^3$, $\vx$ is the point of evaluation and $\vy$ is a point on the boundary and $\vr = \vx - \vy$. 
Recall that $\vn$ is the outward pointing unit normal at $\vy$ to the domain boundary $\Gamma$.
We denote the single layer kernel, also known as the \textit{fundamental solution} or \textit{Green's function} of the \pde, by $S$ and the double layer kernel by $D$.
\begin{enumerate}
  \item \textit{Laplace equation}:
    \begin{align*}
      &\Delta u = 0\\
      &S(\vx,\vy) = \frac{1}{4\pi}\frac{1}{\|\vr\|}, \quad 
      D(\vx, \vy) = - \frac{1}{4\pi} \frac{\vr \cdot \mathbf{n}}{\|\vr\|^3}
    \end{align*}
  \item \textit{Stokes equation}:
    \begin{align*}
      &\mu\Delta u - \nabla p = 0, \,\, \nabla \cdot u = 0\\
      &S(\vx,\vy) = \frac{1}{8\pi\mu}\left( \frac{1}{\|\vr\|} + \frac{\vr \otimes\vr}{\|\vr\|^3}\right), \quad 
      D(\vx, \vy) = - \frac{3}{4\mu\pi} \frac{\vr \otimes\vr}{\|\vr\|^5}(\vr \cdot \mathbf{n})
    \end{align*}
  \item \textit{Elasticity equation}:
    \begin{align*}
      &\mu\Delta u - \frac{\mu}{1-2\nu}\nabla(\nabla \cdot u) = 0\\
      &S(\vx,\vy) = \frac{1}{16\pi\mu(1-\nu)}\left( \frac{3-4\nu}{\|\vr\|} + \frac{\vr \otimes\vr}{\|\vr\|^3}\right), \\
      &D(\vx, \vy) = - \frac{1-2\nu}{8\mu(1-\nu)} \left(
      \frac{1}{\|\vr\|^3} \left(
      \vr \otimes \vn - (\vr\cdot\vn) I - \vn \otimes \vr 
      \right) - \frac{3}{1-2\nu}
      \frac{(\vr\cdot\vn) (\vr \otimes \vr)}{\|\vr\|^5}
      \right)
    \end{align*}
\end{enumerate}


\section{Computing the closest point on a patch  \label{app:closest_point}}
%\subsection{Optimization\label{app:closest_point_opt}}
%\begin{figure}%[!htb]
%  \centering
%    \includegraphics[width=.5\linewidth]{figs/newton-opt.pdf}
%    \mcaption{fig:newton-opt}{Closest point optimization schematic}{}
%\end{figure}
We include our algorithm to find the closest point $\vy$ on a patch $\vP$ to a point $\vx \in \mathbb{R}^3$ in the section for completeness.
For a surface or quadrature patch $\vP$ and point $\vx \in \mathbb{R}^3$, 
we need to compute a point $\vy = \vP(s^*, t^*)$ such that
\begin{equation}
  (s^*, t^*) = \argmin_{(s,t) \in [-1,1]^2} \|\vx - \vP(s,t)\|_2^2 =  \argmin_{(s,t) \in [-1,1]^2} \vr(s,t)\cdot \vr(s,t)
\end{equation}
where $ \vr = \vr(s,t) = \vx - \vP(s,t)$; let $g(s,t) = \vr\cdot \vr$.
We first consider the unconstrained problem
\begin{equation}
    (s^*, t^*) = \argmin_{(s,t) \in \mathbb{R}^2} \|\vx - \vP(s,t)\|_2^2  = \argmin_{(s,t) \in \mathbb{R}^2} \psi(s,t) 
\end{equation}
We solve this optimization problem with Newton's method.
The first and second derivatives of $\psi$ can be evaluated efficiently, since they are polynomials of fixed order.
The gradient and Hessian of the objective function are:
\begin{equation}
  \nabla \psi  =
  \begin{pmatrix}
    -\vP_s\cdot \vr \\
    -\vP_t\cdot \vr \\
  \end{pmatrix}, \quad
  %\label{eq:grad-newton} 
  \nabla^2 \psi = 
\begin{pmatrix}
  \vP_s \cdot \vP_s - \vr\cdot \vP_{ss} & \vP_s \cdot \vP_t - \vr\cdot \vP_{st}\\
  \vP_s \cdot \vP_t - \vr\cdot \vP_{st} & \vP_t \cdot \vP_t - \vr\cdot \vP_{tt}  \\
\end{pmatrix}.
  \label{equ:grad-hess-newton}
\end{equation}
The optimality conditions are 
\begin{equation}
\vP_s^* \cdot \vr^* = 0, \quad \vP_t^* \cdot \vr^* = 0, \quad (u,v) = (s^*, t^*).
  \label{eq:kkt}
\end{equation}
at a local optimum $(s^*, t^*)$.

Let $\psi_i = \psi(s_i,t_i)$, where $(s_i,t_i)$ is the value of the solution during the $i$th iteration of Newton's method.
To solve for the descent direction in Newton's method, we need to solve
\begin{equation}
  \nabla^2 \psi_i \, \eta_i = -\nabla \psi_i
  \label{eq:newton-system}
\end{equation}
where $\eta_i = (\Delta s_i,\Delta t_i)$ is the $i$th Newton update to $(s_i,t_i)$ such that
\begin{equation}
  s_{i+1} = \alpha_i\Delta s_i + s_i,\quad
  t_{i+1} = \alpha_i\Delta t_i + t_i
  \label{}
\end{equation}

We use four iterations of a backtracking line search with an Armijo condition to compute the step length $\alpha_i$ to ensure an appropriate size step is taken in case the initial guess is outside the region of quadratic convergence.
We compute the solution $(s^*, t^*)$ by iterating
\begin{equation}
  (s_n,t_n) = (s_{n-1}, t_{n-1}) + \alpha_{n-1} \eta_{n-1}, \quad \text{ while } \vP_s \cdot \vr > \err{opt}, \quad \vP_t \cdot \vr > \err{opt},
  \label{eq:descent_iter}
\end{equation}
until convergence, i.e., $\psi_i\approx \err{opt}$, $\vr \approx \vn(\vy)$.

If $(s^*, t^*) \in (-1,1)^2$, then the solution to the unconstrained problem is also the solution to the constrained problem.
However, if the closest point lies in $\mathbb{R}\setminus [-1,1]^2$, we need to ensure the inequality constraints are satisfied.
Additionally, if $(s^*, t^*)$ is on the boundary of $[-1,1]^2$, either $s^*$ or $t^*$ should be exactly zero; with the optimization scheme above, we can only claim that $|s^*| < \err{opt}$ (similarly for $t^*$).
To address both of these troubles, we can solve a one-dimensional projection of \cref{eq:newton-system} on to the boundary of $[-1,1]^2$.
For example, to find the closest point along the edge $v=0$, the Newton iteration becomes
\begin{equation}
  s_n = s_{n-1} + \alpha_{n-1}\frac{-\vP_s \cdot \vr}{\vP_s\cdot \vP_s - \vr\cdot \vP_{ss}},
  \label{eq:geom-newton-1d}
\end{equation}
where $\vP_s$, $\vP_{ss}$ and $\vr$ are evaluated at $s_{n-1}$.
Since the boundary is composed of $[-1,t], [1,t], [s,-1], [s,1]$ for $s,t\in[-1,1]$, we solve \cref{eq:geom-newton-1d} once for each interval.

This final algorithm to compute the closest point is as follows:
\begin{enumerate}
  \item We solve \cref{eq:newton-system} on an extended parameter domain $[-1-c, 1+c]^2$, and terminate the Newton iteration if $(s_i,t_i)$ walks outside this boundary. 
    If the Newton iteration terminates inside $[-1,1]^2$, then we've found the closest point.
    We typically choose $c = .2$.
  \item  If the solution is outside $[-1,1]^2$, we solve \cref{eq:newton-system} along each component of the boundary of $[-1,1]^2$, also on an extended parameter domain $[-1-c,1+c]$,
    by choosing an initial guess contained within the interval.
    The solution to these four problems that yields a minimal distance to $\vx$ to used as the closest point, if the solution is inside $[-1,1]$.
  \item If the closest point on the boundary is still outside of $[-1,1]^2$, the
      closest point to $\vx$ is chosen from $\vP(-1,-1), \vP(-1,1), \vP(1,-1),$ and $\vP(1,1)$ closest to $\vx$.
\end{enumerate}
This gives us an algorithm to compute the closest point on a quadrature patch $\vP$ to $\vx$.
The \oned and \twod Newton minimizations converge in ten iterations on average.



\lipsum

%% Note: If your thesis has more than one appendix, NYU requires a "list of
%% appendices" page before the body of the thesis. I don't provide the tools
%% to create that here, so you're on your own for that one... Sorry.


%%%% Input bibliography file %%%%%%%%%%%%%%%
%% For computer science dissertations, I'd recommend using the bibly package
%% to automatically create the .bib file from your citations:
%% https://github.com/michael-emmi/bibly

\cleardoublepage
\phantomsection
\bibliographystyle{apalike}
\addcontentsline{toc}{chapter}{Bibliography}

% \bibliography{dblp,references}

% The following is just for the sample template,
% I'd recommend deleting this and using the \bibliography command above
\begin{thebibliography}{99}
\bibitem[Lamport, 1994]{lamport94}
  Leslie Lamport,
  \textit{\LaTeX: a document preparation system},
  Addison Wesley, Massachusetts,
  2nd edition,
  1994.
\end{thebibliography}


\end{document}

%%% Local Variables:
%%% mode: latex
%%% TeX-master: t
%%% End:
