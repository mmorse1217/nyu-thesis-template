% ----------------------------------------
% Packages
% ----------------------------------------

% 
% Place here your \usepackage's. Some recommended packages are already included.
%

% Graphics:
\usepackage[final]{graphicx}
\graphicspath{{figs/}}
%\usepackage{graphicx} % use this line instead of the above to suppress graphics in draft copies
%\usepackage{graphpap} % \defines the \graphpaper command

% Uncomment this to indent first line of each section:
% \usepackage{indentfirst}

% Good AMS stuff:
\usepackage{amsthm} % facilities for theorem-like environments
\usepackage[tbtags]{amsmath} % a lot of good stuff!

% Fonts and symbols:
\usepackage{amsfonts}
\usepackage{amssymb}

% Set the fonts
\RequirePackage[T1]{fontenc}
\ifxetex
  \RequirePackage[tt=false]{libertine}
\else
  \RequirePackage[tt=false, type1=true]{libertine}
\fi
\RequirePackage[varqu]{zi4}
\RequirePackage[libertine]{newtxmath}


% For typesetting inference rules
\usepackage{mathpartir}
% \usepackage{pftools}  % A local package
\newcommand{\bmmax}{2}
\usepackage{bm}

% Formatting tools:
%\usepackage{relsize} % relative font size selection, provides commands \textsmalle, \textlarger
%\usepackage{xspace} % gentle spacing in macros, such as \newcommand{\acims}{\textsc{acim}s\xspace}

% Page formatting utility:
%\usepackage{geometry}

\usepackage{booktabs}   %% For formal tables:
                        %% http://ctan.org/pkg/booktabs
\usepackage[labelformat=simple]{subcaption} %% For complex figures with subfigures/subcaptions
                        %% http://ctan.org/pkg/subcaption
% Options to subcaption are to label and refer to subfigures as Fig 1(a) etc.
\renewcommand\thesubfigure{(\alph{subfigure})}

\usepackage[T1]{fontenc} % needed for scaling fancy fonts (?)
\usepackage[utf8]{inputenc} % not sure this is needed

\usepackage{amssymb}
%\usepackage[table]{xcolor}

% For code
\usepackage[final]{listings}
\lstset{mathescape=true}

% For code highlighting
% \usepackage{bold-extra}

% Tikz
\usepackage{tikz,pgf}
\usetikzlibrary{matrix,arrows,positioning,calc,fit,backgrounds}
\usetikzlibrary{external}
\usetikzlibrary{patterns}
\usepackage{pgfplots,import,tkz-euclide}

% To control enum item labelling/numbering
\usepackage[shortlabels, inline]{enumitem}
% To give custom item labels and reference them
\makeatletter
\newcommand{\myitem}[1][]{
  \protected@edef\@currentlabel{#1}%
\item[#1]
}
\makeatother

% To stop aligned env swallowing up []s
\usepackage{mathtools}

% To use ifstrempty
\usepackage{etoolbox}

% For math mode tables
\usepackage{array}
% A text column in array
\newcolumntype{L}{>$l<$}

% For \llbracket and \rrbracket
\usepackage{stmaryrd}

% For dashed boxes
\usepackage{dashbox}

% For big separating conjunction
\usepackage{scalerel}

% For mathpar environment
\usepackage{mathpartir}

\usepackage{xspace}
\usepackage{multirow}

% To stop citations overflowing lines
\usepackage{breakcites}

% For citet command
\usepackage{natbib}
\setcitestyle{%
    authoryear,%
    open={[},close={]},citesep={;},%
    aysep={},yysep={,},%
    notesep={, }}
\let\cite\citep

%%
%% Place here your \newtheorem's:
%%

\theoremstyle{plain}
\newtheorem{theorem}{Theorem}[chapter]
\newtheorem{conjecture}[theorem]{Conjecture}
\newtheorem{proposition}[theorem]{Proposition}
\newtheorem{lemma}[theorem]{Lemma}
\newtheorem{corollary}[theorem]{Corollary}
\theoremstyle{definition}
\newtheorem{example}[theorem]{Example}
\newtheorem{definition}[theorem]{Definition}
\theoremstyle{plain}


% ----------------------------------------
% Generic definitions
% ----------------------------------------
% Required packages: listings, tikz

% A footnote without a marker
\newcommand\blfootnote[1]{%
  \begingroup
  \renewcommand\thefootnote{}\footnote{#1}%
  \addtocounter{footnote}{-1}%
  \endgroup
}

\renewcommand{\le}{\leqslant}
\renewcommand{\ge}{\geqslant}
% \renewcommand{\emptyset}{\ensuremath{\varnothing}}
% \newcommand{\ds}{\displaystyle}

% Math stuff
\newcommand{\R}{\ensuremath{\mathbb{R}}}
\newcommand{\Q}{\ensuremath{\mathbb{Q}}}
\newcommand{\Z}{\ensuremath{\mathbb{Z}}}
\newcommand{\N}{\ensuremath{\mathbb{N}}}
\newcommand{\T}{\ensuremath{\mathbb{T}}}
\newcommand{\C}{\ensuremath{\mathbb{C}}}
\newcommand{\eps}{\varepsilon}
\newcommand{\closure}[1]{\ensuremath{\overline{#1}}}
%\newcommand{\acim}{\textsc{acim}\xspace}
%\newcommand{\acims}{\textsc{acim}s\xspace}

\newcommand{\Land}{\bigwedge}
\newcommand{\Lor}{\bigvee}
\newcommand{\es}{\emptyset}
\newcommand{\incl}{\subseteq}
\newcommand{\impl}{\Rightarrow}
\renewcommand{\iff}{\Leftrightarrow}
\newcommand{\ra}{\rightarrow}
\newcommand{\sat}{\vDash}
\newcommand{\notsat}{\nvDash}
\newcommand{\proves}{\vdash}
\newcommand{\provesIff}{\mathrel{\dashv\vdash}}
\newcommand{\boolTrue}{\top}
\newcommand{\boolFalse}{\bot}

\newcommand{\dom}{\operatorname{\mathsf{dom}}}
\newcommand{\range}{\operatorname{\mathsf{rng}}}
\newcommand{\restrict}[2]{{#1}|_{#2}}
\newcommand{\pto}{\rightharpoonup}

\newcommand{\defeq}{\coloneqq}
\newcommand{\defiff}{\vcentcolon\iff}

\newcommand{\pipe}{\triangleright}

%% Caligraphic
%\newcommand{\Aa}{{\mathcal{A}}}
%\newcommand{\Bb}{{\mathcal{B}}}
%\newcommand{\Cc}{{\mathcal{C}}}
%\newcommand{\Dd}{{\mathcal{D}}}
%\newcommand{\Ee}{{\mathcal{E}}}
%\newcommand{\Ff}{{\mathcal{F}}}
%\newcommand{\Gg}{{\mathcal{G}}}
%\newcommand{\Hh}{{\mathcal{H}}}
%\newcommand{\Ii}{{\mathcal{I}}}
%\newcommand{\Jj}{{\mathcal{J}}}
%\newcommand{\Kk}{{\mathcal{K}}}
%\newcommand{\Ll}{{\mathcal{L}}}
%\newcommand{\Mm}{{\mathcal{M}}}
%\newcommand{\Nn}{{\mathcal{N}}}
%\newcommand{\Oo}{{\mathcal{O}}}
%\newcommand{\Pp}{{\mathcal{P}}}
%\newcommand{\Qq}{{\mathcal{Q}}}
%\newcommand{\Rr}{{\mathcal{R}}}
%\newcommand{\Ss}{{\mathcal{S}}}
%\newcommand{\Tt}{{\mathcal{T}}}
%\newcommand{\Uu}{{\mathcal{U}}}
%\newcommand{\Vv}{{\mathcal{V}}}
%\newcommand{\Ww}{{\mathcal{W}}}
%\newcommand{\Yy}{{\mathcal{Y}}}
%\newcommand{\Zz}{{\mathcal{Z}}}

% Wrappers: Parens, brackets, etc
% \newcommand{\op}[1]{\operatorname{#1}}
\newcommand{\paren} [1] {\ensuremath{ \left( {#1} \right) }}
\newcommand{\bigparen} [1] {\ensuremath{ \Big( {#1} \Big) }}
% \newcommand{\bracket}[1]{\left[#1\right]}
\newcommand{\tuple}[1]{\ensuremath{\langle #1 \rangle}}
\newcommand{\abs}[1]{\ensuremath{\lvert #1 \rvert}}
% \newcommand{\set}[1]{\ensuremath{\left\{#1\right\}}}
\newcommand{\setcomp}[2]{\ensuremath{\left\{#1\;\middle|\;#2\right\}}}

% References
\newcommand{\refCh}[1]{Chapter~\ref{#1}}
\newcommand{\refSc}[1]{Section~\ref{#1}}
% \newcommand{\refSc}[1]{\S\ref{#1}}
\newcommand{\refFig}[1]{Figure~\ref{#1}}
\newcommand{\refDef}[1]{Definition~\ref{#1}}
\newcommand{\refLem}[1]{Lemma~\ref{#1}}
\newcommand{\refThm}[1]{Theorem~\ref{#1}}
\newcommand{\refAlg}[1]{Algorithm~\ref{#1}}
\newcommand{\refEx}[1]{Example~\ref{#1}}
\newcommand{\refCor}[1]{Corollary~\ref{#1}}
\newcommand{\refTab}[1]{Table~\ref{#1}}
\newcommand{\refEq}[1]{\ensuremath{(\ref{#1})}}
\newcommand{\refRule}[1]{(\ref{#1})}
\newcommand{\refApp}[1]{Appendix~\ref{#1}}

\newcommand{\tool}[1]{\textsf{#1}}
\newcommand{\code}[1]{\textnormal{\small\texttt{#1}}}
% \newcommand{\code}[1]{\text{\lstinline{#1}}}

% TODO have macros for \forall and \exists

\newcommand{\tick}{\ensuremath{\checkmark}}
\newcommand{\cross}{\text{\sffamily X}}


% ----------------------------------------
% Paper specific macros & commands
% ----------------------------------------

%% hedgehog defs
\usepackage[capitalize]{cleveref}
\usepackage[linesnumbered,lined,ruled,vlined,commentsnumbered]{algorithm2e}

\usepackage{enumitem}

\newlist{criteria}{enumerate}{10}
\setlist[criteria]{label*=\arabic*}

\newtheorem{heuristic}[theorem]{Heuristic}
\crefname{heuristic}{heuristic}{heuristics}
\Crefname{heuristic}{Heuristic}{Heuristics}

\crefname{criteriai}{Criterion}{Criteria}
\Crefname{criteriai}{Criterion}{Criteria}
\crefname{alg}{algorithm}{algorithms}
\Crefname{alg}{Algorithm}{Algorithms}
\crefname{algocf}{Algorithm}{Algorithms} %cref refers to a counter
% Put your definitions here
% From Abtin's template
\usepackage{subcaption}
\usepackage{xifthen}
\newcommand{\mcaption}[3]{
        \ifthenelse{\isempty{#2}}
              {\caption[#1]{\small{#3} \label{#1}}}
              {\caption[#2]{{\sc #2.} \small{#3} \label{#1}}}
        }
\newcommand{\algcaption}[3]{
        \ifthenelse{\isempty{#3}}
                   {\caption[#2]{{\sc #2.} \label{#1}}}
                   {\caption[#2]{{\sc #2.} \newline\small{#3} \label{#1}}}
        }

%% bloodflow defs
\def\rbc{\abbrev{RBC}\xspace}
\def\cc{\abbrev{CC}\xspace}
\def\rbcs{\abbrev{RBC}s\xspace}
\def\cpu{\abbrev{CPU}\xspace}
\def\gpu{\abbrev{GPU}\xspace}
\def\mpi{\abbrev{MPI}\xspace}
\def\li{\abbrev{LI}\xspace}
\def\gi{\abbrev{GI}\xspace}
\def\gir{\abbrev{RGI}\xspace}
\def\lic{\abbrev{CLI}\xspace}
\def\lcp{\abbrev{LCP}\xspace}
\def\lcps{\abbrev{LCP}s\xspace}
\def\ncp{\abbrev{NCP}\xspace}
\def\stiv{\abbrev{STIV}\xspace}
\def\p4est{\texttt{p4est}\xspace}
\def\id{\abbrev{ID}\xspace}
\def\ib{\abbrev{IB}\xspace}
\def\lb{\abbrev{LB}\xspace}
\def\dpd{\abbrev{DPD}\xspace}
\def\sph{\abbrev{SPH}\xspace}
\def\skx{\abbrev{SKX}\xspace}
\def\knl{\abbrev{KNL}\xspace}
%\def\sdc{\abbrev{SDC}\xspace}
\newcommand\sdc[1][]{\abbrev{SDC\ifthenelse{\isempty{#1}}{}{\kern 1pt}#1}\xspace}
\def\emdash/{\kern 0.2em---\kern 0.2em}

\newcommand\para[1]{\paragraph*{#1}}
\newcommand\lh[1]{\emph{#1}}           % line heading, useful for itemized

\newcommand\mathbfsf[1]{\bm{\mathsf{#1}}}
\newcommand\vectord[1]{\mathbfsf{#1}}

\newcommand\vB{\vector{B}}
\newcommand\vxd{\vectord{x}}
\newcommand\vXd{\vectord{X}}
\newcommand\vY{\vector{Y}}
\newcommand\vYd{\vectord{Y}}
\newcommand\va{\vector{a}}
\newcommand\vb{\vector{b}}
\newcommand\vf{\vector{f}}
\newcommand\vfd{\vectord{f}}
\newcommand\vF{\vectord{F}}

\newcommand\dt{\ensuremath{\Delta t}}

\newcommand\slyr{single-layer\xspace}
\newcommand\dlyr{double-layer\xspace}
\newcommand\ns{near-singular\xspace}
\newcommand\gl{Gauss--Legendre\xspace}
\newcommand\bc{boundary condition\xspace}

\definecolor{clr1}{RGB}{255, 246, 39}
\definecolor{clr2}{RGB}{124, 22, 28}
\definecolor{clr3}{RGB}{84, 170, 25}
\definecolor{clr4}{RGB}{137, 230, 251}
\definecolor{clr5}{RGB}{226, 49, 39} %red
\definecolor{clr6}{RGB}{12, 59, 136}
\definecolor{clr7}{RGB}{53, 120, 120}
\definecolor{clr8}{RGB}{50, 49, 70}
\definecolor{clr9}{RGB}{255, 0, 255}
\definecolor{clr10}{RGB}{0, 0, 255}
\definecolor{clr11}{RGB}{255, 122, 122}

\definecolor{clr12}{RGB}{130, 130, 130}
\definecolor{clr13}{RGB}{180, 180, 180}
\definecolor{clr14}{RGB}{230, 230, 230}
\definecolor{plt-blue}{rgb}{0.0078,0.2980,0.7961}%blue
\definecolor{plt-orange}{rgb}{1.0000,0.6431,0.2627}%orange
\definecolor{plt-purple}{rgb}{1.0000,0.2863,0.5255}%purple
\definecolor{plt-violet}{rgb}{0.6118,0.1765,1.0000}%violet

\usepackage[export]{adjustbox}  % adjustment boxes in floats
\pgfplotsset{compat=newest}
\usepgfplotslibrary{fillbetween}
\usetikzlibrary{arrows.meta}
\usetikzlibrary{backgrounds}
\usetikzlibrary{pgfplots.groupplots}
\usetikzlibrary{plotmarks}

\newif\ifPlotTikz
\PlotTikztrue
%
\ifPlotTikz
\pgfplotsset{compat=newest}
\usepgfplotslibrary{fillbetween}
\usetikzlibrary{arrows.meta}
\usetikzlibrary{backgrounds}
\usetikzlibrary{pgfplots.groupplots}
\usetikzlibrary{plotmarks}

\pgfplotsset{plot coordinates/math parser=false}
\pgfkeys{/pgf/images/include external/.code=\includegraphics{#1}}
\tikzexternalize[prefix=figs/]
%\tikzset{external/export=false}%disable externalization

\newcommand{\includepgf}[2][1]{
\beginpgfgraphicnamed{#2}%
\tikzsetnextfilename{external-#2}%
\scalebox{#1}{\subimport{figs/}{#2.pgf}}%
\endpgfgraphicnamed%
}

%%disable hyperref in tikz figures
\makeatletter
\tikzset{
    every picture/.style={
        execute at begin picture={
            \let\ref\@refstar
        }
    }
}
\makeatother
%
\else%for ifPlotTikz
%
\newcommand{\includepgf}[2][1]{
\scalebox{#1}{\includegraphics[]{external-#2}}%
}
%
\fi%for ifPlotTikz
%%% Local Variables:
%%% mode: latex
%%% TeX-master: "thesis"
%%% End:
