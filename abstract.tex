% Replace this with your abstract:

Numerical simulation of red blood cell flows through capillaries is a important investigational tool in the biological sciences.
The ability to rapidly evaluate the impact of vessel and cell geometries, plasma viscosity, and particulate densities on macroscopic physiology is crucial to pursuing further biological understanding. 
Experimental techniques are costly and time-consuming, while analytical approaches are often of limited practical use in realistic scenarios, ultimately underscoring the importance of numerical methods.

In this work, we construct such a simulation, capable of simulating microliters of blood flowing through realistic vasculature.
Due to the micrometer length scales of typical capillaries, we can model the blood plasma as a Stokesian fluid and red blood cells as inextensible, deformable membranes.
By reformulating the viscous flow as a set of boundary integral equations, we are able to produce a method that has optimal complexity with high-order accuracy that is capable of handling dense particulate suspensions in complex geometries.

The blood flow simulation relies on a robust solver for elliptic partial differential equations, applied to Stokes flow.
A core component of the solver is a novel fast algorithm to compute the value of the solution near and on the domain boundary, known as \qbkix.
We provide a set of algorithms to guarantee the accuracy of \qbkix on piecewise smooth surfaces, discuss the error behavior and complexity of \qbkix, and evaluate its performance.

Leveraging this solver in a confined blood flow simulation involves advecting deformable particulates along the flow trajectory. 
Large timesteps are required for an efficient simulation, but can cause collisions among cells and with the vessel wall if performed naively. 
We present collision detection and resolution algorithms for the red blood cells and the blood vessel.
We parallelize \qbkix and the collision algorithms and scale the final simulation to nearly 35,000 cores.

%%% Local Variables:
%%% mode: latex
%%% TeX-master: "thesis"
%%% End:
